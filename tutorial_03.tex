\documentclass{article}
\usepackage{amsfonts, amssymb, amsmath}
\usepackage{float}
\parindent 0px
\pagestyle{empty}


% Everything before begin{document} is called PREAMBLE.

\begin{document}
The distributive property states that $a(b+c)=ab+ac$, for all $a, b, c \in \mathbb{R}$. \\[6pt]
The equivalence class of $a$ is $[a]$. \\[6pt]
The set $A$ is defined to be $\{1,2,3\}$. \\[6pt]
The movie ticket costs $\$11.50$.

$$2(\frac{1}{x^2-1})$$
$$2\left(\frac{1}{x^2-1}\right)$$
$$2\left[\frac{1}{x^2-1}\right]$$
$$2\left\{\frac{1}{x^2-1}\right\}$$ % remember to put backslash to use curly brackets {}
$$2\left \langle    \frac{1}{x^2-1} \right     \rangle $$
$$\left | \frac{1}{x^2-1}  \right |$$

$$\left. \frac{dy}{dx} \right|_{x=1}$$ % to not show any symbol with either left or right just type a period '.'

$$1+{\left ( \frac{1}{1+x} \right )}$$

$$\left ( \frac{1}{1+{\left ( \frac{1}{1+x} \right )}} \right )$$

Tables: \\

\begin{tabular}{|c||c|c|c|c|c|} % this means that six columns, and all with center aligned and seperated by pipe character (|), the first and second column will be seperated with two | chars
\hline
$x$ & 1 & 2 & 3 & 4 & 5 \\ \hline
$f(x)$ & 10 & 11 & 12 & 13 & 14 \\ \hline
\end{tabular}

\vspace{1cm}

% In LaTex, the compiler automatically decided the best place for your table.
% This below table will not be shown at the place its code is written.
\begin{table}
\begin{tabular}{|c||c|c|c|c|c|}
\hline
$x$ & 1 & 2 & 3 & 4 & 5 \\ \hline
$f(x)$ & $\frac{1}{2}$ & 11 & 12 & 13 & 14 \\ \hline
\end{tabular} 
\end{table}

% This table will be shown at the place its code is written due to the [H] (this requires float package)
\begin{table}[H]
\centering % to center the table
\def\arraystretch{1.2} % padding
\begin{tabular}{|c||c|c|c|c|c|}
\hline
$x$ & 1 & 2 & 3 & 4 & 5 \\ \hline
$f(x)$ & $\frac{1}{2}$ & 11 & 12 & 13 & 14 \\ \hline
\end{tabular} 
\caption{These values represent the function $f(x)$}
\end{table}


\begin{table}[H]
\centering % to center the table
\caption{The relationship between $f$ and $f'$}
\def\arraystretch{1.2} % padding
\begin{tabular}{|c|c|}
\hline
$f(x)$ & $f'(x)$ \\ \hline
$x > 0$ & The function $f(x)$ is increasing. \\ \hline
\end{tabular} 
\end{table}


\begin{table}[H]
\centering % to center the table
\caption{The relationship between $f$ and $f'$}
\def\arraystretch{1.2} % padding
\begin{tabular}{|c|c|}
\hline
$f(x)$ & $f'(x)$ \\ \hline
$x > 0$ & The function $f(x)$ is increasing. The function $f(x)$ is increasing. The function $f(x)$ is increasing. \\ \hline
\end{tabular} 
\end{table}


\begin{table}[H]
\centering % to center the table
\caption{The relationship between $f$ and $f'$}
\def\arraystretch{1.2} % padding
\begin{tabular}{|l|p{3in}|} % p stands for paragraph aligned.
\hline
$f(x)$ & $f'(x)$ \\ \hline
$x > 0$ & The function $f(x)$ is increasing. The function $f(x)$ is increasing. The function $f(x)$ is increasing. \\ \hline
\end{tabular} 
\end{table}


Arrays:

\begin{align}
% Everything in align is in math-mode
% compiler ignores spaces in math-mode
% to type normal text in math-mode use \text command
5x^2 \text{place your words here.}\\
5x^2 \, \text{place your words here.}
\end{align}

\begin{align}
5x^2-9 = x+3\\
5x^2-x = x+12
\end{align}

% &= aligns the equations according to the '='
\begin{align*} % align* doesn't numbers the equations
5x^2-9 &= x+3\\
5x^2-x &= x+12\\
&=12+x-5x^2
\end{align*}


\begin{align}
5x^2-9 &= x+3\\
5x^2-x &= x+12\\
&=12+x-5x^2
\end{align}


\end{document}
